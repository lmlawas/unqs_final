\documentclass[journal]{IEEE/IEEEtran}
\usepackage{pkg/pgf-pie}
\usepackage[T1]{fontenc}
\usepackage{inconsolata}
\usepackage{graphicx}

\graphicspath{ {images/} }

\usepackage{color}

\definecolor{pblue}{rgb}{0.13,0.13,1}
\definecolor{pgreen}{rgb}{0,0.5,0}
\definecolor{pred}{rgb}{0.9,0,0}
\definecolor{pgrey}{rgb}{0.46,0.45,0.48}
\usepackage{cite,graphicx,pgfplots,amsmath}
\usepackage{listings}
\usepackage{blindtext}
\lstset{language=Java,
  showspaces=false,
  showtabs=false,
  breaklines=true,
  showstringspaces=false,
  breakatwhitespace=true,
  commentstyle=\color{pgreen},
  keywordstyle=\color{pblue},
  stringstyle=\color{pred},
  basicstyle=\ttfamily,
%  moredelim=[il][\textcolor{pgrey}]{$$},
  moredelim=[is][\textcolor{pgrey}]{\%\%}{\%\%}
}



\newcommand{\SPTITLE}{UPLB Network Queue Simulator (UNQS): Analyzing Network Performance For Internet Bandwidth Management}
\newcommand{\ADVISEE}{Leensey M. Lawas}
\newcommand{\ADVISER}{Danilo J. Mercado}

\newcommand{\BSCS}{Bachelor of Science in Computer Science}
\newcommand{\ICS}{Institute of Computer Science}
\newcommand{\UPLB}{University of the Philippines Los Ba\~{n}os}
\newcommand{\REMARK}{\thanks{Presented to the Faculty of the \ICS, \UPLB\
                             in partial fulfillment of the requirements
                             for the Degree of \BSCS}}
        
\markboth{CMSC 200 Undergraduate Thesis, \ICS}{}
\title{\SPTITLE}
\author{\ADVISEE~and~\ADVISER%
\REMARK
}
\pubid{\copyright~2016~ICS \UPLB}

%%%%%%%%%%%%%%%%%%%%%%%%%%%%%%%%%%%%%%%%%%%%%%%%%%%%%%%%%%%%%%%%%%%%%%%%%%

\begin{document}

% TITLE
\maketitle

% ABSTRACT
\begin{abstract}
Given the increasing demand for bandwidth, UNQS was developed to simulate real traffic data and to identify the optimal bandwidth setting as measured by duration (seconds), throughput (bits per second), and flow loss (percentage). The results show that \textbf{TBA}.
\end{abstract}

% INDEX TERMS
\begin{keywords}
bandwidth management, internet, network performance, network simulation, queueing, traffic engineering
\end{keywords}

% INTRODUCTION
\section{Introduction}
With the prevalence of internet usage in this digital age, the rise of demand for fast and reliable execution of online services is inevitable. Whether it is for personal use, like video chatting with friends and family from abroad, or for commercial and business transactions, customers want to make sure their services are done efficiently without slowing down or timing out. Client requests are sent simultaneously that when the server responses (or traffic) are returned, the routers are unable to inspect long fields in Internet Protocol (IP) packet headers quickly and are unable to reassemble and segment packets fast enough, causing performance bottleneck \cite{pazos_gerla_rigolio_1999}. A quick solution to the problem would be increasing bandwidth size, because as the demand for services increases, the bandwidth must also be increased \cite{communication_news_2001}. However, this method is costly and inefficient, which is why traffic engineering (TE) takes place. 

Awduche, Chiu, Elwalid, Widjaja, and Xiao of The Internet Society (2002) \cite{awduche_chiu_elwalid_xiao_2002} defined that internet traffic engineering deals with evaluating network performance and optimizing it. Bandwidth is the unit of measurement, usually in Kbps or Mbps, used to monitor network performance for quantifying how much information a communication channel can handle \cite{teach_ict_nd}.

\subsection{Background of the study}
Tong and Yang (2007)\cite{tong_yang_2007} cited that there have been many studies on TE, but most of them dealt with route selection algorithms, and few tackled bandwidth management techniques. For this study, bandwidth management is the TE method chosen. Kanu, Kuyoro, Ogunlere, \& Adegbenjo \cite{kanu_kuyoro_ogunlere_adegbenjo_2012} define bandwidth management as an optimization technique that helps differentiate the types of network traffic from each other and determine which client or service should be prioritized. In short, bandwidth management allocates the available bandwidth depending on network traffic and client/service priority.

In an article named \textit{Bandwidth management pays off} (2002)\cite{communication_news_2001}, two key devices were identified to help in bandwidth management: traffic shaping or congestion avoidance mechanisms and queueing techniques. \textit{Congestion avoidance mechanisms} or \textit{congestion control} locates where in the router the packets do not enter the system, and finds an alternative route so the packets do not block the way and cause timeout \cite{jacobson_1988}. \textit{Queueing techniques}, on the other hand, help predict and direct the traffic flow by implementing a constraint or constraints to provide the services as demanded \cite{gross_harris_1974}). In addition, queueing network models are known for accuracy and efficiency \cite{lazowska_zahorjan_graham_sevcik_1984}.

\subsection{Significance of the study}
Inefficient internet bandwidth management can lead to dissatisfied customers and reduced productivity. As long there is a need for “high quality of corporate customer satisfaction”, bandwidth management will continue to grow as a body of knowledge \cite{duzbeck_2006}. This study simulated actual traffic data within the University of the Philippines Los Banos (UPLB) Network in order to determine whether the existing bandwidth is optimal or not. Additionally, the results can also be used for future planning that can entail significant cost reductions, thus optimizing both bandwidth and budget to provide quality service.

\subsection{Objectives of the study}
The general objective of the study is to efficiently simulate the UPLB network traffic by identifying the most optimal bandwidth setting. Specifically, the study was able to:

\begin{enumerate}
\item Collect traffic data from the UPLB network;
\item Simulate the traffic data using various bandwidth sizes;
\item Take note of the duration, throughput, and flow loss for each simulation; and
\item Determine the most optimal network setting using graphs and simple statistics.
\end{enumerate}

\clearpage

\subsection{Time and Place of the study}
The study was conducted from January 2017 until November 2017, at the Institute of Computer Science, UPLB.

\subsection{Scope and Limitation of the study}
The study is limited to monitoring and simulating a portion of the UPLB network. Also, it focused on the traditional queueing technique known as First In-First Out Queueing (FIFO).

In determining the most optimal bandwidth, the throughput, latency, and flow loss values will be measured, noted, and compared. \textit{Throughput} is the number of tasks accomplished over a period of time, \textit{latency} is the time it takes for a fixed task to be finished \cite{martin_roth_nd}, and \textit{flow loss} is the percentage of dropped flows over the total number of flows.

\section{Review of Related Literature}
Several studies have been conducted which attempted to manage internet bandwidth as efficiently as possible. With a goal to provide speedy transaction of certain services such as e-mailing, video streaming, downloading, and many more, internet bandwidth management plays an important role not just for business and commerce applications, but as well as personal usage, for customer satisfaction and improved network performance. Because of the many details enumerated, the demand for internet bandwidth management has never been greater.

From 2015 to 2016, Filipino Internet users had increased from an estimated 42.3\% to 43.5\% \cite{internet_live_stats_2016}. The implications of this statistics to the UPLB academe can also be applicable, as more students and workers enter the university to make use of the campus network. The increase entails a growth in demand for larger internet bandwidth. With a number of users sending multiple requests for different services with varying sizes, data traffic becomes congested. No end-user wants delays, slow downs, or timeouts, in accomplishing the services they requested. Instead, end-users want fast execution of their requests so they can proceed to doing other tasks.

To fix the problem, bandwidth management takes place. Instead of paying for an increase in bandwidth for a temporary fix \cite{communication_news_2001}, bandwidth management aims to properly allocate the already existing bandwidth size as effectively and as efficiently as possible.

Before packets are received by the destination address, they first arrive in packet switches. These packet switches are in charge of queueing the packets and forwarding them eventually to the destination \cite{comer_1999}. Thus enters the scheduling algorithms used to identify which packets must be distrbuted first to the computers in the network.

Traffic classification is an initial stage that plays a key role in some scheduling algorithms, such as Priority Queueing (PQ). Because of bandwidth contraints, traffic classification helps in managing the fixed, limited, and available bandwidth. Classification can be payload-based, meaning a field of the payload is examined and used for classification \cite[Chapter~5]{cisco_2008}. An early traffic classification technique \cite{schneider_1996} made use of port numbers, which worked best for well-known or reserved ports. The other method for classifcation uses statistical analysis of traffic behavior \cite[Chapter~5]{cisco_2008}.

A proposed priority packet scheduling algorithm \cite{karim_2012} made use of three priority queues that gave importance to real-time traffic (priority 1) over non-real time traffic (priorities 2 and 3). Its result suggested of a better performance opposed to FIFO and multi-level queue scheduler algorithms.

From a different study, packet scheduling algorithms using fair queueing and two additional variants were simulated to compare the delay. It showed that Weighted Fair Queueing (WFQ) and Self Clock Fair Queueing (SCFQ) experienced a linear delay, whereas the Worst Case Weighted Fair Queueing (WF2Q) share the output link \cite{muhilan_2013}.

The aforementioned studies inspired and influenced this study, which used traffic data as input for simulating the FIFO scheduling algorithm using various bandwidth settings.% classified by port number as input for the simulations of FIFO, PQ, and WFQ scheduling algorithms.

% MATERIALS AND METHODS
\section{Methodology}
\subsection{Traffic Collection}
With assistance from ITC, a mirror port was setup and connected to a 64-bit Ubuntu server named as \texttt{babage}. The traffic monitoring application, \texttt{ntopng}, was installed to the server. \texttt{MySQL} database management system was also installed, which shall contain the database where traffic flow data from \texttt{ntong} will be dumped. To run \texttt{ntopng}, a configuration file needs to be set to identify the network interface(s) and network(s) to be monitored, the database and table to be dumped at, and the HTTP port where the web portal can be accessed. Data was collected from November 13 to 20. Using the \texttt{mysqldump} tool, the .sql file was generated for the researcher to have a copy of the database outside of the UPLB network. 

\texttt{ntopng} uses the Unix timestamp to label the arrival time of packets into the switch (known as \texttt{FIRST\_SWITCHED}) and their exit time from switch to their destination hosts (\texttt{LAST\_SWITCHED}). This timestamp counts the seconds that have passed since January 1, 1970 (Coordinated Universal Time/UTC).

\subsection{Simulation}

A class diagram (section \ref{fig:uml}, pg. \pageref{fig:uml}) was constructed to get an overview of how UNQS was implemented in the Java programming language. The following items give a brief description of each class/interface.

\begin{figure*}
\includegraphics[width=\textwidth]{UNQS_UML}
\label{fig:uml}\caption{\textbf{UNQS Class Diagram}}
\end{figure*}

\begin{enumerate}
\item \textbf{Cofiguration.java} - This class is responsible for setting, updating, validating, and displaying the configuration for the database connection and simulation settings.

\item \textbf{Flow.java} - A flow is a network link between a source host and a destination host. For the purpose of this program, the IP addresses of both hosts are not defined as this class' attribute.

\item \textbf{NetworkBuffer.java} - This class was made for readability purposes and is a collection of flows.

\item \textbf{Schedule.java} - Schedule is an interface that has final-static-defined variables \textit{FIFO} (0), \textit{PQ} (1), and \textit{WFQ} (2). All of its methods are abstract and therefore must be overriden by the classes that will implement it.

\item \textbf{FirstInFirstOut.java} - FirstInFirstOut implements the \texttt{Schedule} interface and overrides its methods in order to perform FIFO's logic.

\item \textbf{UNQS.java} - This contains the main function where the connection to the database is established. Time is counted from the configuration's defined range of \texttt{start\_time} to \texttt{end\_time} and shall continue to iterate until all flows within the defined network buffer(s) are scheduled from the switch to their destination.
\end{enumerate}

\section{Results and Discussion}

\subsection{Traffic Description}

Network traffic was collected from November 13-20. Specifically, the earliest interaction was sent at 1510502646 and the last flow arrived at 1511168758.

The total number of flows was 554,392, the largest flow having 1,362,515,708 bytes of data, the average size being 517,998.49 bytes and the smallest containing 60 bytes.

The most active source IP address, 2886731784, sent  the most data, accounting for 5.4324\% of the total flows while the IP address 4026531834 received the most interactions, thus describing the 9.9321\%.

\begin{table*}[ht]
\centering
\caption{\textbf{Top 9 Out-Flows}}
\label{top-out-flows}
\begin{tabular}{|c|c|c|}
\hline
\textbf{DESTINATION (\# of flows)} 		& \textbf{TOTAL BYTES} 	& \textbf{\%} \\ \hline
	KDDI CORPORATION (1)				&  62409079786			& 60.6        \\ \hline
	MULTICAST (3)						&  12310903701          & 11.9        \\ \hline
	Google LLC(3)						&  5202456530           & 5.0         \\ \hline
	Facebook, Inc. (1)					&  2771346197           & 2.7         \\ \hline	
	Apple Inc.(1)						&  1054547100           & 1.0         \\ \hline
\end{tabular}
\end{table*}

\begin{table*}[ht]
\centering
\caption{\textbf{Top 9 In-Flows}}
\label{top-in-flows}
\begin{tabular}{|c|c|c|}
\hline
\textbf{SOURCE (\# of flows)} 		& \textbf{TOTAL BYTES} 	& \textbf{\%} \\ \hline
    KDDI CORPORATION(1)				&  79135454093			& 43.0        \\ \hline
    Google LLC(3)					&  27958817762          & 15.2        \\ \hline
    WorldStream B.V.(1)				&  3426880570           & 1.9         \\ \hline
    M247 Ltd(2)						&  4286472241           & 2.3         \\ \hline
    Converge ICT Solutions Inc.(2)	&  3967550023			& 2.2         \\ \hline    
\end{tabular}
\end{table*}

\begin{table*}[ht]
\centering
\caption{\textbf{Top 10 Destination Ports}}
\label{top-dest-ports}
\begin{tabular}{|c|c|c|c|}
\hline
\textbf{PORT NUMBER}			& \textbf{PROTOCOL}			& \textbf{TOTAL FLOWS}		& \textbf{\%} 	 \\ \hline
    443							& HTTPS						&  65088					& 11.7  	     \\ \hline
    53							& DNS						&  60765        			& 11.0			 \\ \hline
    1900						& SSDP						&  52579					& 9.5			 \\ \hline
    5060						& SIP						&  43101          			& 7.8   	     \\ \hline
    445							& Microsoft-DS				&  28746           			& 5.2        	 \\ \hline
    80							& HTTP						&  18485					& 3.3         	 \\ \hline
    5355						& LLMNR						&  17090    		        & 3.1    	     \\ \hline
    0							& Reserved					&  15732		            & 2.8       	 \\ \hline
    7437						& Faximum					&  13734        		    & 2.5         	 \\ \hline
    161							& SNMP						&  9142        			    & 1.6         	 \\ \hline
    67							& Bootstrap Protocol Server	&  8602        			    & 1.6         	 \\ \hline
\end{tabular}
\end{table*}

% IP addresses
The top destination IP addresses for out-flow traffic shown in Table \ref{top-out-flows} and the top source IP addresses for in-flow traffic seen in Table \ref{top-in-flows} belong to the companies that are listed as follows:

\begin{enumerate}
\item \textbf{KDDI Corporation} is a telecommunications business based in Japan. It provides content hosting over optimized networks, and ICT and business services and solutions.

\item \textbf{Google LLC} is an American multinational technology company that specializes in Internet-related services and products such as online advertising technologies, search engine, cloud computing, software, and hardware.

\item \textbf{WorldStream B.V.} is a popular Internet Service Provider based in the Netherlands and is used by customers from all over the world. It provides cost-effective services to secure hosting environment, and offers hardware and Operating System technologies.

\item \textbf{M247 Ltd} is a UK-based technology company that offers  services and tools to secure network and data while providing connectivity and internet infastructure that expands to a global scale.

\item \textbf{Converge ICT} is a Philippine technology company with the fastest growing fiber internet and offers services to ensure pure end-to-end fiber internet connection, thus reducing data loss, faster speed and bandwidth.

\item \textbf{Facebook} is an American online social media and social networking service company.

\item \textbf{Apple Inc.} is an American multinational technology company that designs, develops, and sells consumer electronics, computer software, and online services. Multicast IPs allow group communication to be sent simultaneously to multiple computers.

\item \textbf{Multicast} * is not a company, but rather a form of data transmission that allows multiple hosts to receive data simultaneously.

\end{enumerate}

% references
%http://www.kddi.com/english/corporate/kddi/our-business/ may 28 2018
%https://www.worldstream.nl/en/about/company
%https://m247.com/about-us/
%http://www.convergeict.com/about-us/
%https://tools.ietf.org/html/rfc1112

% Ports

The Internet Assigned Numbers Authority (IANA) is responsible for associating port numbers with certain internet protocols used by network applications. These identifications can be found in IANA's \textit{Service Name and Transport Protocol Port Number Registry}. Table \ref{top-dest-ports} listed the top destination port numbers as recorded in the data.

% https://www.iana.org/assignments/service-names-port-numbers/service-names-port-numbers.txt

\begin{enumerate}

\item \textbf{Hypertext Transfer Protocol Secure (HTTPS)} - secures communication over a network

\item \textbf{Domain Name System (DNS)} - matches names to IP addresses and vice versa to facilitate network communications

\item \textbf{Simple Service Discovery Protocol (SSDP)} - advertises presence information to locate available network services

\item \textbf{Session Inititation Protocol (SIP)} - signals and controls multimedia communication sessions in Voice over IP (VoIP) applications like online voice and video calls

\item \textbf{Microsoft-Directory Services (MS-DS)} - follows the Server Message Block (SMB) protocol for shared access to files, printers, and serial ports and other communications between nodes on a network

\item \textbf{Hypertext Transfer Protocol (HTTP)} - facilitates data communication for the World Wide Web

\item \textbf{Link Local Multicast Name Resolution} - is based on DNS and allows both IPv4 and IPv6 hosts to perform name resolution for hosts on the same local links

\item \textbf{0 (Reserved)} and \textbf{7437 (Faximum)} - have no specifics protocols linked to them, and therefore are likely abused ports to send computer attacks and harmful computer and network content like viruses

\item \textbf{Simple Network Management Protocol (SNMP)} - collects and organizes information about managed devices on IP networks, and can modify information to change device behavior

\item \textbf{Bootstrap Protocol Server (BOOTP)} - exclusive protocol for IPv4; Dynamic Host Configuration Protocol (DHCP) server receives requests upon booting the client's computer

\end{enumerate}

%\linebreak
%\hrule

\subsection{Data Analysis}

Three network performance metrics were accounted for as a result of the simulation, namely, duration, throughput, and flow loss. They are described in the items listed below.

\begin{enumerate}
\item \textbf{Duration}


Duration \textit{d} is a function of the arrival time \textit{t\textsubscript{0}} and last switched time \textit{t\textsubscript{n}} seen as follows:

\[
	d(t_0, t_n) = t_n - t_0
\]

\item \textbf{Throughput}

Throughput \textit{tput} was computed as 
\[
	tput = \frac{count(p_s)}{d}
\]

where \textit{ts} is the total size (in bytes) that was successfully switched, and \textit{d} as the computed duration. 

\item \textbf{Flow Loss}

The last network performance parametric looked at is the flow loss \textit{l}, solved by

\[
	l = \frac{count(f_l)}{count(f_l + f_s)} \times 100
\]

where all dropped flows \textit{f\textsubscript{l}} are counted and divided by the total number of flows \textit{f\textsubscript{l}} plus \textit{f\textsubscript{s}} (switched flows) multiplied by 100, to get the percentage. 

\end{enumerate}

Due to the limited time and resources, the actual data was unable to finish running under the latest implementation of UNQS. Two very small trials were attempted and compared to the result of an SQL query to the database.
\section{Conclusion and Recommendation}

\subsection{Old Results}
\begin{enumerate}
  \item \textbf{Queueing techniques}. Other queueing techniques aside from the native FIFO scheduling may be implemented with the help of the Schedule interface.
  \item \textbf{Traffic classification}. In the original implementation, the port numbers were used for identifying the priority of the packets under the PQ scheduling algorithm. Use of other traffic classification methods, such as statistical method or machine learning classification techniques may improve the effect of prioritizing traffic in the performance of PQ and other queueing techniques which may make use of traffic classification.
\end{enumerate}

% BIBLIOGRAPHY
\bibliographystyle{./IEEE/IEEEtran}
\bibliography{./cs190-ieee}
% \nocite{*}

% BIOGRAPHY
\begin{biography}[{\includegraphics{./yourPicture.eps}}]{\ADVISEE}
She is a BS Computer Science undergraduate student. She not only writes code, but also songs, poems, and stories.
\end{biography}

\end{document}
